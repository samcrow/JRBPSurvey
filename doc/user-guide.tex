\documentclass{article}
\usepackage{geometry}
\geometry{a4paper}

% ISO 8601
\usepackage[yyyymmdd]{datetime}
\renewcommand{\dateseparator}{--}

\title{JRBP Survey User Guide}
\author{Sam Crow}
\date{\today}

\begin{document}
\maketitle{}

\section{Supported devices}

JRPB Survey will run on any Android device running Android 4.0 or later. For full functionality, the device should be able to find its location using GPS and communicate using Wi-Fi.

If the device has a compass sensor, JRPB Survey will display a compass.

\section{Usage}

\subsection{Navigating to a site}

Scroll and zoom the map to find the site that you would like to navigate to. Tap on the site in the map view to select it. A circle will appear around the site, and a line will appear leading from your location to the site.

\subsection{Entering data}

Once you have tapped on a site to select it, you can make an observation at that site. Tap the data entry button in the action bar to start recording an observation.

Check the box for each ant species that you find around the site. If you see any species that does not have a check box, describe it in the notes field.

Tap the save button in the action bar to save the observation. The observation will be saved locally and then uploaded when the device has internet access.

\paragraph{Correcting mistakes}

If you submit an incorrect observation, go back and make another observation for the same site. Enter the correct information, and write in the notes that the previous observation was wrong. Someone will delete the incorrect observation.

\end{document}